\ifx\allfiles\undefined
\documentclass[12pt, a4paper, oneside, UTF8]{ctexbook}
\def\configPath{../config}
\def\basicPath{\configPath/basic}


% 在这里定义需要的包
\usepackage{amsmath}
\usepackage{amsthm}
\usepackage{graphicx}
\usepackage{mathrsfs}
\usepackage{pifont}     % 特殊符号支持
\usepackage{amssymb} % 用于 \textdbend 符号
\usepackage{enumitem}
\usepackage{geometry}
\usepackage[colorlinks, linkcolor=black]{hyperref}
\usepackage{stackengine}
\usepackage{yhmath}
\usepackage{extarrows}
\usepackage{tcolorbox} % 引入tcolorbox包
\usepackage{xcolor} % for colors
\usepackage{amssymb} % for \textdbend

\usepackage{fancyhdr}
\usepackage[dvipsnames, svgnames]{xcolor}
\usepackage{listings}
\usepackage{titlesec}

\usepackage[strict]{changepage} 
\usepackage{framed}
%\usepackage{color}
\usepackage{mathrsfs}
\usepackage{cleveref}
\usepackage{multicol}

\input{\basicPath/custom}
% 定义了一些常用的颜色,这些颜色将在后续的环境中使用。
\definecolor{greenshade}{rgb}{0.90,1,0.92}
\definecolor{redshade}{rgb}{1.00,0.88,0.88}
\definecolor{brownshade}{rgb}{0.99,0.95,0.9}
\definecolor{lilacshade}{rgb}{0.95,0.93,0.98}
\definecolor{orangeshade}{rgb}{1.00,0.88,0.82}
\definecolor{lightblueshade}{rgb}{0.8,0.92,1}
\definecolor{purple}{rgb}{0.81,0.85,1}
\definecolor{shadecolor}{RGB}{241, 241, 255}

\definecolor{structurecolor}{RGB}{0, 102, 204}
% 这里定义了 'second' 颜色(你可以根据需要调整颜色值)
\definecolor{second}{rgb}{0.6, 0.2, 0.2}  % 定义 'second' 颜色为红棕色调

% 设置定理环境的整体风格为“定义”风格。这通常意味着定理主体使用正常字体,而非斜体。
\theoremstyle{definition}
% 自定义彩色圆圈数字命令
\newcommand{\circled}[1]{\textcolor{blue}{\textcircled{\textcolor{black}{\small #1}}}}
%\newcommand{\circled}[1]{\textcircled{\small #1}}

% 重新定义 notes 环境
\newenvironment{notes}{
	\par\noindent\makebox[0pt][r]{%
		\scriptsize\color{red!90}\textdbend\quad}% 图标
	\textbf{\color{red!90}Note:} \normalfont % 红色提示标题,正体字体
}{\par}

% 定义 change 环境,使用圆圈数字标注每一个变化项
%\newenvironment{change}{
%	\begin{enumerate}[label=\small\protect\circled{\arabic*}]}{
%\end{enumerate}}

\newenvironment{change}{
	\begin{enumerate}[label=\hfill\circled{\arabic*}, itemsep=1em]
	}{
	\end{enumerate}
}
% 定义 change2 环境
\newenvironment{change2}{
	\begin{tcolorbox}[colback=white, colframe=gray, boxrule=0.5mm, sharp corners=south]
		\begin{enumerate}[label=\centering\circled{\arabic*}]
		}{
		\end{enumerate}
	\end{tcolorbox}
}
% 定义 tip 环境,用于显示提示信息
\newenvironment{tip}{
	\par\noindent\makebox[-3pt][r]{%
		\scriptsize\color{green!90}\ding{43}\quad}% 绿色图标
	\textbf{\color{green!90}Tip:} \itshape % 绿色的提示标题
}{\par}

% 定义 warning 环境,用于显示警告信息
\newenvironment{warning}{
	\par\noindent\makebox[-3pt][r]{%
		\scriptsize\color{orange!90}\ding{70}\quad}% 橙色图标
	\textbf{\color{orange!90}Warning:} \itshape % 橙色的警告标题
}{\par}

% 定义 important 环境,用于显示重要信息
\newenvironment{important}{
	\par\noindent\makebox[-3pt][r]{%
		\scriptsize\color{red!90}\ding{72}\quad}% 红色图标
	\textbf{\color{red!90}Important:} \itshape % 红色的重点标题
}{\par}

% 定义 examplezero 环境,并提供自动编号
\newcounter{examplezero}[section]
\renewcommand{\theexamplezero}{\thesection.\arabic{examplezero}}

\newenvironment{examplezero}[1][]{
	\refstepcounter{examplezero} % 增加 examplezero 计数器
	\par\noindent\textbf{\color{purple!90}Example \theexamplezero:} \itshape #1 \rmfamily}{\par}
	
% 定义introduction环境
\newenvironment{introduction}[1][Introduction]{
	\begin{tcolorbox}[title={#1}]
		\begin{multicols}{2}
			\begin{itemize}[label=\textcolor{structurecolor}{\upshape\scriptsize$\bullet$}]  % 使用$\bullet$
			}{
			\end{itemize}
		\end{multicols}
	\end{tcolorbox}
}



% 这里给出了不同环境的自用定义形式,创建了一系列自定义的定理环境,每个环境都有自己的编号方式和名称。
\newtheorem{myDefn}{\indent Definition(定义)}[section] % 定义
\newtheorem{myLemma}{\indent Lemma(引理)}[section] % 引理
\newtheorem{myThm}[myLemma]{\indent Theorem(定理)} % 定理
\newtheorem{myCorollary}[myLemma]{\indent Corollary(推论)} % 推论
\newtheorem{myCriterion}[myLemma]{\indent Criterion(标准)} % 标准
\newtheorem*{myRemark}{\indent Remark(备注)} % 备注
\newtheorem{myProposition}{\indent Proposition(命题)}[section] % 命题

% formal 环境用于创建带有彩色边框和背景的框架,包含两个颜色参数
\newenvironment{formal}[2][]{%
	\def\FrameCommand{%
		\hspace{1pt}%
		{\color{#1}\vrule width 2pt}%
		{\color{#2}\vrule width 4pt}%
		\colorbox{#2}%
	}%
	\MakeFramed{\advance\hsize-\width\FrameRestore}%
	\noindent\hspace{-4.55pt}%
	\begin{adjustwidth}{}{7pt}\vspace{2pt}\vspace{2pt}
	}
	{%
		\vspace{2pt}\end{adjustwidth}\endMakeFramed%
}

% 定义不同的自定义定理环境
\newenvironment{defn}{%
	\begin{formal}[Green]{greenshade}\vspace{-\baselineskip / 2}\begin{myDefn}}%
		{\end{myDefn}\end{formal}}

\newenvironment{thm}{%
	\begin{formal}[LightSkyBlue]{lightblueshade}\vspace{-\baselineskip / 2}\begin{myThm}}%
		{\end{myThm}\end{formal}}

\newenvironment{lemma}{%
	\begin{formal}[Plum]{lilacshade}\vspace{-\baselineskip / 2}\begin{myLemma}}%
		{\end{myLemma}\end{formal}}

\newenvironment{corollary}{%
	\begin{formal}[BurlyWood]{brownshade}\vspace{-\baselineskip / 2}\begin{myCorollary}}%
		{\end{myCorollary}\end{formal}}

\newenvironment{criterion}{%
	\begin{formal}[DarkOrange]{orangeshade}\vspace{-\baselineskip / 2}\begin{myCriterion}}%
		{\end{myCriterion}\end{formal}}

\newenvironment{rmk}{%
	\begin{formal}[LightCoral]{redshade}\vspace{-\baselineskip / 2}\begin{myRemark}}%
		{\end{myRemark}\end{formal}}

\newenvironment{proposition}{%
	\begin{formal}[RoyalPurple]{purple}\vspace{-\baselineskip / 2}\begin{myProposition}}%
		{\end{myProposition}\end{formal}}

% 创建一个新计数器 problem,按章节编号
\newcounter{problem}[chapter] 
\newenvironment{problem}{%
	\stepcounter{problem}% 增加problem计数器的值
	\begin{shaded}%
		\par\noindent\textbf{题目 \thechapter.\theproblem}%
	}{%
	\end{shaded}%
	\par%
}
% 定义 answer 环境
\newenvironment{answer}{\par\noindent\textbf{证明 }}{\par}

% 定义 example 定理环境,不与 examplezero 冲突
\newtheorem{example}{\indent \color{SeaGreen}{Example}}[section]

% 修改 proofname 为自定义样式
\renewcommand{\proofname}{\indent\textbf{\textcolor{TealBlue}{Proof}}}

% 定义 solution 环境,作为定理环境的变种
\newenvironment{solution}{%
	\begin{proof}[\indent\textbf{\textcolor{TealBlue}{Solution}}]}{\end{proof}}
\input{\basicPath/format}

\begin{document}
	\else
	\fi
	\chapter{例子}
	
	\section{举例说明}
	
	% 定义
	\begin{defn}
		设 $X$ 是一个集合,$d: X \times X \to \mathbb{R}$ 是一个满足以下条件的函数:
		\begin{enumerate}
			\item 对任意 $x, y \in X$,有 $d(x, y) \geq 0$,且 $d(x, y) = 0$ 当且仅当 $x = y$。
			\item 对任意 $x, y \in X$,有 $d(x, y) = d(y, x)$。
			\item 对任意 $x, y, z \in X$,有 $d(x, z) \leq d(x, y) + d(y, z)$。
		\end{enumerate}
		则称 $d$ 为 $X$ 上的一个\textbf{度量},$X$ 称为一个\textbf{度量空间}。
	\end{defn}
	
	% 定理
	\begin{thm}
		在度量空间 $(X, d)$ 中,任意闭球都是完备的,当且仅当 $X$ 本身是完备的\cite{choy20194ds}。
	\end{thm}
	
	% 引理
	\begin{lemma}
		如果一个序列在度量空间中是柯西序列,则该序列在该空间中有极限。
	\end{lemma}
	
	% 推论
	\begin{corollary}
		每个完备的度量空间都是巴拿赫空间。
	\end{corollary}
	
	% 标准
	\begin{criterion}
		一个度量空间是完备的,当且仅当每个柯西序列在该空间中收敛。
	\end{criterion}
	
	% 备注
	\begin{rmk}
		注意,完备性是度量空间中一个非常重要的性质,它在分析中起着关键作用。例如,实数集 $\mathbb{R}$ 是完备的,而有理数集 $\mathbb{Q}$ 则不是。
	\end{rmk}
	
	% 命题
	\begin{proposition}
		在任何度量空间中,有限维子空间都是完备的。
	\end{proposition}
	
	% 示例
	\begin{example}
		考虑实数集 $\mathbb{R}$ 上的标准度量 $d(x, y) = |x - y|$。这是一个完备的度量空间,因为每个柯西序列在 $\mathbb{R}$ 中都有极限。
	\end{example}
	
	% 问题与解答
	\begin{problem}
		证明在 $\mathbb{R}^n$ 中,任何两个不同的点之间的距离都是正的。
	\end{problem}
	
	\begin{answer}
		设 $x = (x_1, x_2, \ldots, x_n)$ 和 $y = (y_1, y_2, \ldots, y_n)$ 是 $\mathbb{R}^n$ 中的两个不同点。则至少存在一个索引 $i$,使得 $x_i \neq y_i$。根据欧几里得距离的定义:
		\[
		d(x, y) = \sqrt{(x_1 - y_1)^2 + (x_2 - y_2)^2 + \cdots + (x_n - y_n)^2} > 0
		\]
		因为至少有一个 $(x_i - y_i)^2 > 0$。
	\end{answer}
	
	% 解决方案示例
	\begin{solution}
		要证明任意两个不同点之间的距离为正,假设 $x \neq y$。这意味着存在至少一个坐标 $i$,使得 $x_i \neq y_i$。因此,
		\[
		(x_i - y_i)^2 > 0
		\]
		由于所有平方项都是非负的,且至少有一个严格正的项,因此整个距离的平方和是正的,故距离 $d(x, y) > 0$。
	\end{solution}
	
	
	\begin{examplezero}[这是一个简单的例子]
		在这个例子中,我们展示了如何定义并使用一个名为 \texttt{examplezero} 的环境。这个环境会自动编号,并允许我们添加一些描述。
	\end{examplezero}
	
	\begin{examplezero}
		这是另一个没有标题的例子。LaTeX 可以自动为例子编号。
	\end{examplezero}
	
	
	\begin{example}
		这是一个定理类的 \texttt{example} 环境示例,使用了 \texttt{theorem} 风格。
	\end{example}


	\begin{change}
		\item 第一个变化项。
		\item 第二个变化项。
		\item 第三个变化项。
	\end{change}
	
	
	\begin{change2}
		\item 第一个变化项。
		\item 第二个变化项。
		\item 第三个变化项。
	\end{change2}
	

	
	\begin{tip}
		这是一个使用 tip 环境的示例,它带有一个绿色的图标和提示信息。
	\end{tip}
	
	
	\begin{warning}
		这是一个使用 warning 环境的示例,它带有一个橙色的图标和警告信息。
	\end{warning}

	
	\begin{important}
		这是一个使用 important 环境的示例,它带有一个红色的图标和重要信息。
	\end{important}
	\begin{example}
		求微分方程$y''-2y'-3y=3x+1$的一个特解.
	\end{example}
	\begin{solution}
		这是二阶常系数非齐次线性微分方程,且函数$f(x)$是$e^{\lambda{x}}P_m(x)$型,其中
		\[
		\lambda = 0,\;P_m(x) = 3x+1
		\]
		与所给方程对应的齐次方程为
		\[
		y''-2y'-3y=0
		\]
		其特征方程为
		\[
		r^2-2r-3 = 0
		\]
		由于$\lambda = 0$不是特征方程的根,所以设特解
		\[
		y* = b_0 x + b_1
		\]
		带入所给方程,得
		\[
		-3b_0 x - 2b_0 - 3b_1 = 3x+1
		\]
		比较等式两端$x$同次幂的系数,易得$b_0 = -1,\;b_1 = \dfrac{1}{3}$,于是求得一个特解为
		\[
		y* = -x + \frac{1}{3}
		\]
	\end{solution}
	\begin{introduction}
		\item 这是第一项。
		\item 这是第二项。
		\item 这是第三项。
	\end{introduction}
	\begin{solution}
		hxdh
	\end{solution}
	
	\section{多元函数}
	\begin{defn}
		设二元函数$f(P) = f(x,\,y)$的定义域为$D$,点$P_0(x_0,\,y_0)$是$D$的聚点,如果存在常数$A$,对于任意给定正数$\varepsilon$,总存在正整数$\delta$,使得当点$P(x,\,y) \in D \cap \mathring{U}(P_0,\,\delta)$时,都有
		\[
		|f(P)-A| = |f(x,\,y) - A| < \varepsilon
		\]
		成立,那么就称常数$A$为函数$f(x,\,y)$当$(x,\,y) \to (x_0,\,y_0)$时的极限(二重极限),记作
		\[
		\lim_{(x,\,y) \to (x_0,\,y_0)}f(x,\,y) = A \quad \lor \quad \lim_{P \to P_0}f(P) = A
		\]
	\end{defn}
	
	\myspace{1}
	
	任意一点$P \in \R^2$与任意一个点集$E \subset \R^2$之间有以下三种关系的一种:
	\begin{itemize}[leftmargin=45pt]
		\item \textbf{内点}:如果存在点$P$的某个邻域$U(P)$,使得$U(P) \subset E$,那么称$P$为$E$的内点.
		\item \textbf{外点}:如果存在点$P$的某个邻域$U(P)$,使得$U(P) \cap E = \varnothing$,那么称$P$为$E$的外点.
		\item \textbf{边界点}:如果点$P$在任意邻域内既含有属于$E$的点,又含有不属于$E$的点,那么称$P$为$E$的边界点.
	\end{itemize}
	
	\begin{problem}
		这是一道简单的题目
	\end{problem}
	
	\begin{problem}
		这是一道简单的题目2
	\end{problem}
	\begin{answer}
		这是一个解答
	\end{answer}
	
	\begin{proof}
		这是一个证明的示例。
	\end{proof}
	
	\begin{solution}
		这是解决方案的详细描述。
	\end{solution}

$\mathcal{A B C U}$ 

$\mathscr{ABCdef u U}$
	\ifx\allfiles\undefined
\end{document}
\fi
