% 这里给出了不同环境的自用定义形式,创建了一系列自定义的定理环境,每个环境都有自己的编号方式和名称。
\newtheorem{myDefn}{\indent Definition(定义)}[section] % 定义
\newtheorem{myLemma}{\indent Lemma(引理)}[section] % 引理
\newtheorem{myThm}[myLemma]{\indent Theorem(定理)} % 定理
\newtheorem{myCorollary}[myLemma]{\indent Corollary(推论)} % 推论
\newtheorem{myCriterion}[myLemma]{\indent Criterion(标准)} % 标准
\newtheorem*{myRemark}{\indent Remark(备注)} % 备注
\newtheorem{myProposition}{\indent Proposition(命题)}[section] % 命题

% formal 环境用于创建带有彩色边框和背景的框架,包含两个颜色参数
\newenvironment{formal}[2][]{%
	\def\FrameCommand{%
		\hspace{1pt}%
		{\color{#1}\vrule width 2pt}%
		{\color{#2}\vrule width 4pt}%
		\colorbox{#2}%
	}%
	\MakeFramed{\advance\hsize-\width\FrameRestore}%
	\noindent\hspace{-4.55pt}%
	\begin{adjustwidth}{}{7pt}\vspace{2pt}\vspace{2pt}
	}
	{%
		\vspace{2pt}\end{adjustwidth}\endMakeFramed%
}

% 定义不同的自定义定理环境
\newenvironment{defn}{%
	\begin{formal}[Green]{greenshade}\vspace{-\baselineskip / 2}\begin{myDefn}}%
		{\end{myDefn}\end{formal}}

\newenvironment{thm}{%
	\begin{formal}[LightSkyBlue]{lightblueshade}\vspace{-\baselineskip / 2}\begin{myThm}}%
		{\end{myThm}\end{formal}}

\newenvironment{lemma}{%
	\begin{formal}[Plum]{lilacshade}\vspace{-\baselineskip / 2}\begin{myLemma}}%
		{\end{myLemma}\end{formal}}

\newenvironment{corollary}{%
	\begin{formal}[BurlyWood]{brownshade}\vspace{-\baselineskip / 2}\begin{myCorollary}}%
		{\end{myCorollary}\end{formal}}

\newenvironment{criterion}{%
	\begin{formal}[DarkOrange]{orangeshade}\vspace{-\baselineskip / 2}\begin{myCriterion}}%
		{\end{myCriterion}\end{formal}}

\newenvironment{rmk}{%
	\begin{formal}[LightCoral]{redshade}\vspace{-\baselineskip / 2}\begin{myRemark}}%
		{\end{myRemark}\end{formal}}

\newenvironment{proposition}{%
	\begin{formal}[RoyalPurple]{purple}\vspace{-\baselineskip / 2}\begin{myProposition}}%
		{\end{myProposition}\end{formal}}

% 创建一个新计数器 problem,按章节编号
\newcounter{problem}[chapter] 
\newenvironment{problem}{%
	\stepcounter{problem}% 增加problem计数器的值
	\begin{shaded}%
		\par\noindent\textbf{题目 \thechapter.\theproblem}%
	}{%
	\end{shaded}%
	\par%
}
% 定义 answer 环境
\newenvironment{answer}{\par\noindent\textbf{证明 }}{\par}

% 定义 example 定理环境,不与 examplezero 冲突
\newtheorem{example}{\indent \color{SeaGreen}{Example}}[section]

% 修改 proofname 为自定义样式
\renewcommand{\proofname}{\indent\textbf{\textcolor{TealBlue}{Proof}}}

% 定义 solution 环境,作为定理环境的变种
\newenvironment{solution}{%
	\begin{proof}[\indent\textbf{\textcolor{TealBlue}{Solution}}]}{\end{proof}}