\ifx\allfiles\undefined
\documentclass[12pt, a4paper, oneside, UTF8]{ctexbook}
\def\configPath{../config}
\def\basicPath{\configPath/basic}


% 在这里定义需要的包
\usepackage{amsmath}
\usepackage{amsthm}
\usepackage{graphicx}
\usepackage{mathrsfs}
\usepackage{pifont}     % 特殊符号支持
\usepackage{amssymb} % 用于 \textdbend 符号
\usepackage{enumitem}
\usepackage{geometry}
\usepackage[colorlinks, linkcolor=black]{hyperref}
\usepackage{stackengine}
\usepackage{yhmath}
\usepackage{extarrows}
\usepackage{tcolorbox} % 引入tcolorbox包
\usepackage{xcolor} % for colors
\usepackage{amssymb} % for \textdbend

\usepackage{fancyhdr}
\usepackage[dvipsnames, svgnames]{xcolor}
\usepackage{listings}
\usepackage{titlesec}

\usepackage[strict]{changepage} 
\usepackage{framed}
%\usepackage{color}
\usepackage{mathrsfs}
\usepackage{cleveref}
\usepackage{multicol}

\input{\basicPath/custom}
% 定义了一些常用的颜色,这些颜色将在后续的环境中使用。
\definecolor{greenshade}{rgb}{0.90,1,0.92}
\definecolor{redshade}{rgb}{1.00,0.88,0.88}
\definecolor{brownshade}{rgb}{0.99,0.95,0.9}
\definecolor{lilacshade}{rgb}{0.95,0.93,0.98}
\definecolor{orangeshade}{rgb}{1.00,0.88,0.82}
\definecolor{lightblueshade}{rgb}{0.8,0.92,1}
\definecolor{purple}{rgb}{0.81,0.85,1}
\definecolor{shadecolor}{RGB}{241, 241, 255}

\definecolor{structurecolor}{RGB}{0, 102, 204}
% 这里定义了 'second' 颜色(你可以根据需要调整颜色值)
\definecolor{second}{rgb}{0.6, 0.2, 0.2}  % 定义 'second' 颜色为红棕色调

% 设置定理环境的整体风格为“定义”风格。这通常意味着定理主体使用正常字体,而非斜体。
\theoremstyle{definition}
% 自定义彩色圆圈数字命令
\newcommand{\circled}[1]{\textcolor{blue}{\textcircled{\textcolor{black}{\small #1}}}}
%\newcommand{\circled}[1]{\textcircled{\small #1}}

% 重新定义 notes 环境
\newenvironment{notes}{
	\par\noindent\makebox[0pt][r]{%
		\scriptsize\color{red!90}\textdbend\quad}% 图标
	\textbf{\color{red!90}Note:} \normalfont % 红色提示标题,正体字体
}{\par}

% 定义 change 环境,使用圆圈数字标注每一个变化项
%\newenvironment{change}{
%	\begin{enumerate}[label=\small\protect\circled{\arabic*}]}{
%\end{enumerate}}

\newenvironment{change}{
	\begin{enumerate}[label=\hfill\circled{\arabic*}, itemsep=1em]
	}{
	\end{enumerate}
}
% 定义 change2 环境
\newenvironment{change2}{
	\begin{tcolorbox}[colback=white, colframe=gray, boxrule=0.5mm, sharp corners=south]
		\begin{enumerate}[label=\centering\circled{\arabic*}]
		}{
		\end{enumerate}
	\end{tcolorbox}
}
% 定义 tip 环境,用于显示提示信息
\newenvironment{tip}{
	\par\noindent\makebox[-3pt][r]{%
		\scriptsize\color{green!90}\ding{43}\quad}% 绿色图标
	\textbf{\color{green!90}Tip:} \itshape % 绿色的提示标题
}{\par}

% 定义 warning 环境,用于显示警告信息
\newenvironment{warning}{
	\par\noindent\makebox[-3pt][r]{%
		\scriptsize\color{orange!90}\ding{70}\quad}% 橙色图标
	\textbf{\color{orange!90}Warning:} \itshape % 橙色的警告标题
}{\par}

% 定义 important 环境,用于显示重要信息
\newenvironment{important}{
	\par\noindent\makebox[-3pt][r]{%
		\scriptsize\color{red!90}\ding{72}\quad}% 红色图标
	\textbf{\color{red!90}Important:} \itshape % 红色的重点标题
}{\par}

% 定义 examplezero 环境,并提供自动编号
\newcounter{examplezero}[section]
\renewcommand{\theexamplezero}{\thesection.\arabic{examplezero}}

\newenvironment{examplezero}[1][]{
	\refstepcounter{examplezero} % 增加 examplezero 计数器
	\par\noindent\textbf{\color{purple!90}Example \theexamplezero:} \itshape #1 \rmfamily}{\par}
	
% 定义introduction环境
\newenvironment{introduction}[1][Introduction]{
	\begin{tcolorbox}[title={#1}]
		\begin{multicols}{2}
			\begin{itemize}[label=\textcolor{structurecolor}{\upshape\scriptsize$\bullet$}]  % 使用$\bullet$
			}{
			\end{itemize}
		\end{multicols}
	\end{tcolorbox}
}



% 这里给出了不同环境的自用定义形式,创建了一系列自定义的定理环境,每个环境都有自己的编号方式和名称。
\newtheorem{myDefn}{\indent Definition(定义)}[section] % 定义
\newtheorem{myLemma}{\indent Lemma(引理)}[section] % 引理
\newtheorem{myThm}[myLemma]{\indent Theorem(定理)} % 定理
\newtheorem{myCorollary}[myLemma]{\indent Corollary(推论)} % 推论
\newtheorem{myCriterion}[myLemma]{\indent Criterion(标准)} % 标准
\newtheorem*{myRemark}{\indent Remark(备注)} % 备注
\newtheorem{myProposition}{\indent Proposition(命题)}[section] % 命题

% formal 环境用于创建带有彩色边框和背景的框架,包含两个颜色参数
\newenvironment{formal}[2][]{%
	\def\FrameCommand{%
		\hspace{1pt}%
		{\color{#1}\vrule width 2pt}%
		{\color{#2}\vrule width 4pt}%
		\colorbox{#2}%
	}%
	\MakeFramed{\advance\hsize-\width\FrameRestore}%
	\noindent\hspace{-4.55pt}%
	\begin{adjustwidth}{}{7pt}\vspace{2pt}\vspace{2pt}
	}
	{%
		\vspace{2pt}\end{adjustwidth}\endMakeFramed%
}

% 定义不同的自定义定理环境
\newenvironment{defn}{%
	\begin{formal}[Green]{greenshade}\vspace{-\baselineskip / 2}\begin{myDefn}}%
		{\end{myDefn}\end{formal}}

\newenvironment{thm}{%
	\begin{formal}[LightSkyBlue]{lightblueshade}\vspace{-\baselineskip / 2}\begin{myThm}}%
		{\end{myThm}\end{formal}}

\newenvironment{lemma}{%
	\begin{formal}[Plum]{lilacshade}\vspace{-\baselineskip / 2}\begin{myLemma}}%
		{\end{myLemma}\end{formal}}

\newenvironment{corollary}{%
	\begin{formal}[BurlyWood]{brownshade}\vspace{-\baselineskip / 2}\begin{myCorollary}}%
		{\end{myCorollary}\end{formal}}

\newenvironment{criterion}{%
	\begin{formal}[DarkOrange]{orangeshade}\vspace{-\baselineskip / 2}\begin{myCriterion}}%
		{\end{myCriterion}\end{formal}}

\newenvironment{rmk}{%
	\begin{formal}[LightCoral]{redshade}\vspace{-\baselineskip / 2}\begin{myRemark}}%
		{\end{myRemark}\end{formal}}

\newenvironment{proposition}{%
	\begin{formal}[RoyalPurple]{purple}\vspace{-\baselineskip / 2}\begin{myProposition}}%
		{\end{myProposition}\end{formal}}

% 创建一个新计数器 problem,按章节编号
\newcounter{problem}[chapter] 
\newenvironment{problem}{%
	\stepcounter{problem}% 增加problem计数器的值
	\begin{shaded}%
		\par\noindent\textbf{题目 \thechapter.\theproblem}%
	}{%
	\end{shaded}%
	\par%
}
% 定义 answer 环境
\newenvironment{answer}{\par\noindent\textbf{证明 }}{\par}

% 定义 example 定理环境,不与 examplezero 冲突
\newtheorem{example}{\indent \color{SeaGreen}{Example}}[section]

% 修改 proofname 为自定义样式
\renewcommand{\proofname}{\indent\textbf{\textcolor{TealBlue}{Proof}}}

% 定义 solution 环境,作为定理环境的变种
\newenvironment{solution}{%
	\begin{proof}[\indent\textbf{\textcolor{TealBlue}{Solution}}]}{\end{proof}}
\input{\basicPath/format}

\begin{document}
\else
\fi

\chapter{实分析}
\section*{第一小部分:一元实变量函数的Lebesgue积分}
\section*{集合、映射与关系的预备知识}

\section{实数集:集合、序列与函数}

\section{Lebesgue测度}


\section{Lebesgue可测函数}
\subsection{可测函数定义及其性质}
\begin{proposition}\label{proposhifenxi1}
	令函数$f$具有可测定义域$E$,则下面叙述等价
	\begin{change}
		\item $\forall c \in R,\{x\in E| f(x)>c\} \in \mathscr{U}$ ,$\mathscr{U}$为可测集族  
		\item $\forall c \in R,\{x\in E| f(x)\geqslant c\} \in \mathscr{U}$
		\item $\forall c \in R,\{x\in E| f(x)<c\} \in \mathscr{U}$
		\item $\forall c \in R,\{x\in E| f(x)\leqslant  c\} \in \mathscr{U}$
	\end{change}
\end{proposition}

\begin{defn}
	定义在$E$上的扩张的实值函数$f$称为是Lebesgue可测的,若它的定义域$E$是可测的且满足\cref{proposhifenxi1}的四个命题之一的。
\end{defn}

\begin{proposition}
	令$f$为可测集合$E$上的实值函数,则函数$f$是可测的当且仅当对每个开集 %$\mathscr{O}$,
	$\mathcal{O}$,$\mathcal{O}$在$f$下的原象为$f^{-1}(\mathcal{O})=\{x \in E|f(x) \in \mathcal{O}\}$是可测的。
\end{proposition}		

\begin{proposition}
	\begin{change}
		\item 在可测集定义域上连续的实值函数是可测的。
		\item 定义在区间上的可测函数是可测的。
		\item 令$f$和$g$为$E$上a.e.有限的可测函数,则线性($\alpha f + \beta g$)和($f\cdot g$)也可测。
		\item  可测函数的复合函数可以不是可测的。
		\item 令$g$为$E$上的可测实值函数,而$f$为定义在整个$R$上的连续实值函数,则复合$f \circ g $是$E$上的可测函数。
		\item 定义具有共同定义域$E$的函数的有限簇$\{ f_k\}_{k=1}^n$,定义$ \text{max}\{f_1,\cdots f_n\} $($\forall x \in E,\text{max}\{f_1,\cdots f_n\}(x)=\text{max}\{f_1(x),\cdots f_n(x)\}$,和上确界函数一致,$\lambda (x)=\sup\limits_{n} f_n(x)$),下确界函数同样定义,可以证明这些函数在$E$上是可测的。
		\item 可以定义一些函数通过max函数,$|f|=\text{max}\{f(x),-f(x)\},f^+=\text{max}\{f(x),0\},f^-=\text{max}\{-f(x),0\}$,若$f$在$E$上可测,则上面的皆可测,并且$f=f^{+}-f^-,|f|=f^{+}+f^-$
	\end{change}
\end{proposition}

	\begin{corollary}
	若具有定义域$E$的函数$f$是可测的,则$|f|$是可测的,且事实上$\forall p >0,|f|^p$在共同定义域$E$上是可测的。
\end{corollary}

\begin{proposition}
	\begin{enumerate}[label=\Roman*]
		 令$f$为$	E$上扩充的实值函数。
		\item  若$f$为可测的,而在$E$上$f=g \ a.e.$,则$g$在$E$上可测。
		\item 对于$E$的可测子集$D$,$f$是可测的当且仅当在$D$和$E\sim D$上的限制是可测的。
	\end{enumerate}
\end{proposition}

\subsection{序列的逐点收敛及其简单逼近}
在数学上,存在有挺多的收敛定义,但其本质都是在某种度量(范数)下趋近于0,比如逐点收敛,均方收敛,依测度收敛(随机上有个以概率1收敛),一致收敛 $\cdots$

\begin{defn}
	对具有公共定义域$E$的函数序列$\{f_n\}$,$E$上的函数$f$,以及$A\subseteq E$
	\begin{enumerate}[label=\Roman*]
		\item $\forall x\in A,\lim\limits_{n \to \infty} f_n = f(x)$,称序列$\{f_n\}$在$A$上逐点收敛于$f$
		\item 在$A$上,$\{f_n\}$a.e.逐点收敛到$f$,等价于$\{f_n\}$在$A\sim B$上逐点收敛,且$m(B)=0$
		\item 序列$\{f_n\}$在$A$上一致收敛,等价于$\forall \epsilon >0,\exists N=N(\epsilon) \in T,\forall n\geq N,\forall x\in A,|f(x)-f_n(x)|<\epsilon$
	\end{enumerate}
\end{defn}
	连续函数的逐点极限不一定连续(一致收敛才能保证),Riemann可积函数的逐点极限不一定是Riemann可积的。
	
\begin{proposition}
	令$\{f_n\}$为$E$上a.e.逐点收敛于$f$的可测函数序列,则$f$为可测函数。
\end{proposition}


\section{Lebesgue积分}



\section{微分与积分}

\section{$L^P$空间:完备性与逼近}



\section{$L^P$空间:对偶与弱收敛}

\ifx\allfiles\undefined
\end{document}
\fi