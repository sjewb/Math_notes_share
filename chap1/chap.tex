\ifx\allfiles\undefined
\documentclass[12pt, a4paper, oneside, UTF8]{ctexbook}
\def\configPath{../config}
\def\basicPath{\configPath/basic}


% 在这里定义需要的包
\usepackage{amsmath}
\usepackage{amsthm}
\usepackage{graphicx}
\usepackage{mathrsfs}
\usepackage{pifont}     % 特殊符号支持
\usepackage{amssymb} % 用于 \textdbend 符号
\usepackage{enumitem}
\usepackage{geometry}
\usepackage[colorlinks, linkcolor=black]{hyperref}
\usepackage{stackengine}
\usepackage{yhmath}
\usepackage{extarrows}
\usepackage{tcolorbox} % 引入tcolorbox包
\usepackage{xcolor} % for colors
\usepackage{amssymb} % for \textdbend

\usepackage{fancyhdr}
\usepackage[dvipsnames, svgnames]{xcolor}
\usepackage{listings}
\usepackage{titlesec}

\usepackage[strict]{changepage} 
\usepackage{framed}
%\usepackage{color}
\usepackage{mathrsfs}
\usepackage{cleveref}
\usepackage{multicol}

\input{\basicPath/custom}
% 定义了一些常用的颜色,这些颜色将在后续的环境中使用。
\definecolor{greenshade}{rgb}{0.90,1,0.92}
\definecolor{redshade}{rgb}{1.00,0.88,0.88}
\definecolor{brownshade}{rgb}{0.99,0.95,0.9}
\definecolor{lilacshade}{rgb}{0.95,0.93,0.98}
\definecolor{orangeshade}{rgb}{1.00,0.88,0.82}
\definecolor{lightblueshade}{rgb}{0.8,0.92,1}
\definecolor{purple}{rgb}{0.81,0.85,1}
\definecolor{shadecolor}{RGB}{241, 241, 255}

\definecolor{structurecolor}{RGB}{0, 102, 204}
% 这里定义了 'second' 颜色(你可以根据需要调整颜色值)
\definecolor{second}{rgb}{0.6, 0.2, 0.2}  % 定义 'second' 颜色为红棕色调

% 设置定理环境的整体风格为“定义”风格。这通常意味着定理主体使用正常字体,而非斜体。
\theoremstyle{definition}
% 自定义彩色圆圈数字命令
\newcommand{\circled}[1]{\textcolor{blue}{\textcircled{\textcolor{black}{\small #1}}}}
%\newcommand{\circled}[1]{\textcircled{\small #1}}

% 重新定义 notes 环境
\newenvironment{notes}{
	\par\noindent\makebox[0pt][r]{%
		\scriptsize\color{red!90}\textdbend\quad}% 图标
	\textbf{\color{red!90}Note:} \normalfont % 红色提示标题,正体字体
}{\par}

% 定义 change 环境,使用圆圈数字标注每一个变化项
%\newenvironment{change}{
%	\begin{enumerate}[label=\small\protect\circled{\arabic*}]}{
%\end{enumerate}}

\newenvironment{change}{
	\begin{enumerate}[label=\hfill\circled{\arabic*}, itemsep=1em]
	}{
	\end{enumerate}
}
% 定义 change2 环境
\newenvironment{change2}{
	\begin{tcolorbox}[colback=white, colframe=gray, boxrule=0.5mm, sharp corners=south]
		\begin{enumerate}[label=\centering\circled{\arabic*}]
		}{
		\end{enumerate}
	\end{tcolorbox}
}
% 定义 tip 环境,用于显示提示信息
\newenvironment{tip}{
	\par\noindent\makebox[-3pt][r]{%
		\scriptsize\color{green!90}\ding{43}\quad}% 绿色图标
	\textbf{\color{green!90}Tip:} \itshape % 绿色的提示标题
}{\par}

% 定义 warning 环境,用于显示警告信息
\newenvironment{warning}{
	\par\noindent\makebox[-3pt][r]{%
		\scriptsize\color{orange!90}\ding{70}\quad}% 橙色图标
	\textbf{\color{orange!90}Warning:} \itshape % 橙色的警告标题
}{\par}

% 定义 important 环境,用于显示重要信息
\newenvironment{important}{
	\par\noindent\makebox[-3pt][r]{%
		\scriptsize\color{red!90}\ding{72}\quad}% 红色图标
	\textbf{\color{red!90}Important:} \itshape % 红色的重点标题
}{\par}

% 定义 examplezero 环境,并提供自动编号
\newcounter{examplezero}[section]
\renewcommand{\theexamplezero}{\thesection.\arabic{examplezero}}

\newenvironment{examplezero}[1][]{
	\refstepcounter{examplezero} % 增加 examplezero 计数器
	\par\noindent\textbf{\color{purple!90}Example \theexamplezero:} \itshape #1 \rmfamily}{\par}
	
% 定义introduction环境
\newenvironment{introduction}[1][Introduction]{
	\begin{tcolorbox}[title={#1}]
		\begin{multicols}{2}
			\begin{itemize}[label=\textcolor{structurecolor}{\upshape\scriptsize$\bullet$}]  % 使用$\bullet$
			}{
			\end{itemize}
		\end{multicols}
	\end{tcolorbox}
}



% 这里给出了不同环境的自用定义形式,创建了一系列自定义的定理环境,每个环境都有自己的编号方式和名称。
\newtheorem{myDefn}{\indent Definition(定义)}[section] % 定义
\newtheorem{myLemma}{\indent Lemma(引理)}[section] % 引理
\newtheorem{myThm}[myLemma]{\indent Theorem(定理)} % 定理
\newtheorem{myCorollary}[myLemma]{\indent Corollary(推论)} % 推论
\newtheorem{myCriterion}[myLemma]{\indent Criterion(标准)} % 标准
\newtheorem*{myRemark}{\indent Remark(备注)} % 备注
\newtheorem{myProposition}{\indent Proposition(命题)}[section] % 命题

% formal 环境用于创建带有彩色边框和背景的框架,包含两个颜色参数
\newenvironment{formal}[2][]{%
	\def\FrameCommand{%
		\hspace{1pt}%
		{\color{#1}\vrule width 2pt}%
		{\color{#2}\vrule width 4pt}%
		\colorbox{#2}%
	}%
	\MakeFramed{\advance\hsize-\width\FrameRestore}%
	\noindent\hspace{-4.55pt}%
	\begin{adjustwidth}{}{7pt}\vspace{2pt}\vspace{2pt}
	}
	{%
		\vspace{2pt}\end{adjustwidth}\endMakeFramed%
}

% 定义不同的自定义定理环境
\newenvironment{defn}{%
	\begin{formal}[Green]{greenshade}\vspace{-\baselineskip / 2}\begin{myDefn}}%
		{\end{myDefn}\end{formal}}

\newenvironment{thm}{%
	\begin{formal}[LightSkyBlue]{lightblueshade}\vspace{-\baselineskip / 2}\begin{myThm}}%
		{\end{myThm}\end{formal}}

\newenvironment{lemma}{%
	\begin{formal}[Plum]{lilacshade}\vspace{-\baselineskip / 2}\begin{myLemma}}%
		{\end{myLemma}\end{formal}}

\newenvironment{corollary}{%
	\begin{formal}[BurlyWood]{brownshade}\vspace{-\baselineskip / 2}\begin{myCorollary}}%
		{\end{myCorollary}\end{formal}}

\newenvironment{criterion}{%
	\begin{formal}[DarkOrange]{orangeshade}\vspace{-\baselineskip / 2}\begin{myCriterion}}%
		{\end{myCriterion}\end{formal}}

\newenvironment{rmk}{%
	\begin{formal}[LightCoral]{redshade}\vspace{-\baselineskip / 2}\begin{myRemark}}%
		{\end{myRemark}\end{formal}}

\newenvironment{proposition}{%
	\begin{formal}[RoyalPurple]{purple}\vspace{-\baselineskip / 2}\begin{myProposition}}%
		{\end{myProposition}\end{formal}}

% 创建一个新计数器 problem,按章节编号
\newcounter{problem}[chapter] 
\newenvironment{problem}{%
	\stepcounter{problem}% 增加problem计数器的值
	\begin{shaded}%
		\par\noindent\textbf{题目 \thechapter.\theproblem}%
	}{%
	\end{shaded}%
	\par%
}
% 定义 answer 环境
\newenvironment{answer}{\par\noindent\textbf{证明 }}{\par}

% 定义 example 定理环境,不与 examplezero 冲突
\newtheorem{example}{\indent \color{SeaGreen}{Example}}[section]

% 修改 proofname 为自定义样式
\renewcommand{\proofname}{\indent\textbf{\textcolor{TealBlue}{Proof}}}

% 定义 solution 环境,作为定理环境的变种
\newenvironment{solution}{%
	\begin{proof}[\indent\textbf{\textcolor{TealBlue}{Solution}}]}{\end{proof}}
\input{\basicPath/format}

\begin{document}
	\else
	\fi
	\part{数学分析}
	\chapter{六大定理的相互证明}
	\section{六大定理}
	
	\label{thm:fubi2}
	\label{pro:max3}
	\label{property:cauchy3}
	\label{tabe222}
	\label{pro:js5}
	
	\begin{introduction}
		\item 确界定理(实数系连续性定理)~\ref{def:int1}
		\item 单调有界定理~\ref{thm:fubi2}
		\item Cauchy定理~\ref{pro:max3}
		\item 区间套定理~\ref{property:cauchy3}
		\item 聚点定理 ~\ref{tabe222}
		\item 有限覆盖定理(Heine-Borel定理)~\ref{pro:js5}
	\end{introduction}
		
\begin{change}
	\item 确界定理(实数系连续性定理\label{def:int1}):非空有上界的实数集必有上确界,非空有下界的实数集必有下确界。
	\item 单调有界定理:在实数系中,单调且有界的数列必定收敛。
	\item Cauchy定理:数列 \(\{x_n\}\) 收敛的充要条件是 \(\{x_n\}\) 是基本(柯西)数列。
	\item 区间套定理:若 \(\{[a_n, b_n]\}\) 形成一个闭区间套,则存在唯一的实数 \(\xi\) 属于所有的闭区间 \([a_n, b_n]\),\(n = 1, 2, 3, \cdots\),且 \( \xi \) 属于所有这些闭区间,并且有 \( \lim_{n \to \infty} b_n \) = \( \lim_{n \to \infty} a_n  \) =\( \xi \) 
	\item 聚点定理、魏尔斯特拉斯定理、致密性定理、Bolzano-Weierstrass定理:
	\begin{itemize}
		\item 聚点定理:在一个有界序列中,至少存在一个聚点。
		\item 魏尔斯特拉斯定理:每个有界数列都有至少一个聚点。
		\item 致密性定理:一个集合是致密的当且仅当它的每一个开覆盖都有有限子覆盖。
		\item Bolzano-Weierstrass定理:任何有界数列必有一个收敛的子序列。
	\end{itemize}
	\begin{enumerate}
		\item \textbf{聚点定理}:
		\begin{itemize}
			\item \textbf{定义}:在一个有界序列中,至少存在一个聚点。
			\item \textbf{说明}:聚点是指序列中某一子序列的极限。如果一个序列是有界的,那么它至少有一个聚点。
			\item \textbf{应用}:该定理说明了有界序列在某种意义上不会“散开”,一定存在某些点是序列的极限点。
		\end{itemize}
		
		\item \textbf{魏尔斯特拉斯定理}(也称为\textbf{有界性原理}):
		\begin{itemize}
			\item \textbf{定义}:每个有界数列都有至少一个聚点。
			\item \textbf{说明}:这个定理与聚点定理基本一致,它强调了有界数列一定存在至少一个聚点。
			\item \textbf{应用}:该定理是序列收敛性分析中的基础,特别是对于有界数列的性质研究。
		\end{itemize}
		
		\item \textbf{致密性定理}:
		\begin{itemize}
			\item \textbf{定义}:一个集合是致密的当且仅当它的每一个开覆盖都有有限子覆盖。
			\item \textbf{说明}:这个定理描述了致密集(通常称为紧致集或紧集)的特性。一个集合如果每一个开覆盖(即用开集覆盖整个集合)都可以找到一个有限的子覆盖,那么这个集合是致密的。
			\item \textbf{应用}:致密性定理在拓扑学中非常重要,用于研究集合的极限性质和连续函数的性质。
		\end{itemize}
		
		\item \textbf{Bolzano-Weierstrass定理}:
		\begin{itemize}
			\item \textbf{定义}:任何有界数列必有一个收敛的子序列。
			\item \textbf{说明}:这个定理指出,有界数列总是可以找到一个收敛的子序列。这意味着有界数列在某种程度上总是可以提取出一个收敛的部分。
			\item \textbf{应用}:Bolzano-Weierstrass定理在分析学中广泛应用,特别是在证明各种关于收敛性的命题时。
		\end{itemize}
	\end{enumerate}
	
	总结:
	\begin{itemize}
		\item \textbf{聚点定理}和\textbf{魏尔斯特拉斯定理}主要关注有界数列的聚点存在性。
		\item \textbf{致密性定理}关注的是集合的覆盖性质,是一个拓扑学定理。
		\item \textbf{Bolzano-Weierstrass定理}进一步说明了有界数列不仅有聚点,而且必有一个收敛的子序列。
	\end{itemize}
	
	它们各自从不同的角度描述了数列和集合的性质,在数学分析和拓扑学中都有着广泛的应用。
	
	\item 有限覆盖定理(Heine-Borel 定理):
	\begin{itemize}
		\item \textbf{定义}:一个集合是紧的当且仅当它是闭且有界的;设 $H$ 为闭区间 $[a,b]$ 的一个(无限)开覆盖,则从 $H$ 中可选出有限个开区间来覆盖 $[a,b]$。
		\item \textbf{说明}:如果一个开覆盖 \( S \) 覆盖了闭区间 \([a, b]\),即 \([a, b] \subseteq \bigcup_{i \in I} U_i\),其中 \( U_i \) 是 \( S \) 中的开区间,则存在有限个开区间 \( U_{i_1}, U_{i_2}, \ldots, U_{i_n} \) 使得 \([a, b] \subseteq \bigcup_{j=1}^n U_{i_j}\)。
	\end{itemize}
	
\end{change}

\begin{change2}
	\item 第一个变化项
	\item 第二个变化项
	\item 第三个变化项
\end{change2}

	\subsection{确界定理}
	
	
	
\section*{确界存在定理 $\Rightarrow$ 有限覆盖定理}


	\ifx\allfiles\undefined
\end{document}
\fi
